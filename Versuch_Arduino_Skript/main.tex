\documentclass[12pt,a4paper]{article}

% Pakete laden
\usepackage{ifthen}    % Für bedingte Anweisungen
\usepackage{xcolor}    % Für farbigen Text
\usepackage{framed}    % Für Rahmen um Text
\usepackage{lipsum}    % Für Beispieltext
\usepackage{environ}   % Für erweiterte Umgebungskontrolle
\usepackage{graphicx}  % Für Bilder
\usepackage[utf8]{inputenc}
\usepackage[ngerman]{babel}
\usepackage{amsmath}
\usepackage{amsfonts}
\usepackage{amssymb}
\usepackage{caption}   % Für Bildunterschriften außerhalb von 'figure'-Umgebungen
\usepackage{float}
\usepackage{cite}
\usepackage{listings}
\usepackage{hyperref}
\usepackage{xcolor}
\usepackage{enumitem}
\usepackage{siunitx}
\sisetup{locale = DE}  % Für deutsche Dezimaltrennzeichen
\usepackage{booktabs}
\usepackage{tabularx}
\usepackage[left=2cm,right=2cm,top=2cm,bottom=2cm]{geometry}
\usepackage{tikz} % Für karierte Boxen

\usepackage{fancyhdr}
\pagestyle{fancy} % Den Seitenstil auf 'fancy' setzen
\fancyhf{} % Alle Kopf- und Fußzeilenfelder bereinigen

% Kopfzeile
\fancyhead[L]{\leftmark} % Links: Kapitelname
\fancyhead[C]{} % Mitte: leer
\fancyhead[R]{\rightmark} % Rechts: Abschnittsname

% Fußzeile
\fancyfoot[L]{Uni Bremen} % Links: Ihr Name oder was auch immer Sie möchten
\fancyfoot[C]{Seite \thepage} % Mitte: Seitenzahl
\fancyfoot[R]{\today} % Rechts: Aktuelles Datum

% Linien oben und unten
\renewcommand{\headrulewidth}{0.4pt} % Linienstärke der Kopfzeile
\renewcommand{\footrulewidth}{0.4pt} % Linienstärke der Fußzeile

% Farben definieren
\definecolor{loesungsblau}{rgb}{0.204, 0.373, 0.667} % Ein dunkles Blau für die Lösungen
\definecolor{aufgabenfarbe}{rgb}{0.85, 0.85, 1} % A lighter shade of blue for the Aufgabenumgebung

% Rahmenstil für Lösungen anpassen
\newenvironment{customframe}{
    \def\FrameCommand{{\color{loesungsblau}\vrule width 4pt \hspace{10pt}}}%
    \MakeFramed {\advance\hsize-\width \FrameRestore}%
}{%
    \endMakeFramed
}

% Variable definieren, um zu steuern, ob Lösungen angezeigt werden sollen
\newboolean{zeigenloesungen}
\setboolean{zeigenloesungen}{true}% Auf 'false' für die Schülerversion

% Lösungsumgebung definieren
\NewEnviron{lösung}{
    \ifthenelse{\boolean{zeigenloesungen}}%
    {%
        \begin{customframe}
        \noindent\textbf{\color{loesungsblau}Lösung:}\\ \BODY
        \end{customframe}%
        \centering
    }%
    {}%
}

% Zähler für Aufgaben definieren, der sich nach Kapiteln richtet
\newcounter{aufgabe}[section]
\renewcommand{\theaufgabe}{\thesection.\arabic{aufgabe}}

% Aufgabenumgebung definieren
\NewEnviron{aufgabe}{
    \refstepcounter{aufgabe} % Aufgabenzähler erhöhen
    \begin{center}
    \begin{minipage}{1\textwidth}
        \colorbox{aufgabenfarbe}{
            \parbox{\dimexpr\textwidth-2\fboxsep}{
                \textbf{Aufgabe \theaufgabe:}\ \BODY % Aufgabennummer anzeigen
            }
        }
    \end{minipage}
    \end{center}
}

% Umgebung für ein kariertes Feld
\newcommand{\karierteBox}[1]{
\ifthenelse{\boolean{zeigenloesungen}}%
{%
}
{%
    \begin{center}
    \begin{tikzpicture}
        \draw[step=0.5, gray, very thin] (0,0) grid (16,#1); % Zeichnet ein 5x#1 Raster
    \end{tikzpicture}
    \end{center}
}
}


\begin{document}

% Titelseite Deckblatt mit Logo

\title{Schüler Workshop - Labor X - Übertragung von Daten mit einem Arduino}
\maketitle
\begin{figure}[H]
    \centering
    \includegraphics[width=0.5\textwidth]{logos/antlogo.pdf}
\end{figure}

\author{}
\date{}

\newpage

\section{Modulationsverfahren}

Das Verfahren, um binäre Daten auf ein analoges Trägersignal zu bringen, wird als digitale Modulation bezeichnet. 
In Abbildung \ref{fig:ook_modulation} sind die Modulationsverfahren ASK und FSK dargestellt. ASK steht für \textit{Amplitude Shift Keying} und FSK für \textit{Frequency Shift Keying}. 
Im ersten Graphen von oben ist das digitale Signal dargestellt, das moduliert werden soll.
Im zweiten Gaphen ist das mit FSK modulierte Signal dargestellt. Der maximale Frequenzunterschied wird dabei als Frequenzhub $\Delta F$ bezeichnet.
Darunter ist das mit ASK modulierte Signal dargestellt. Es handelt sich dabei um sogenanntes \textit{On-Off-Keying} (OOK), bei dem das Signal entweder an oder aus ist.

\begin{aufgabe}
    Bestimmen Sie für Abbildung \ref{fig:ook_modulation} die Trägerfrequenz und den Frequenzhub $\Delta F$ des FSK-Signals.
\end{aufgabe}
\karierteBox{5}

\begin{lösung}
    Durch Ablesen im Graphen bestimmen:

    \[
    f_{0} = \frac{2}{\SI{0,2}{\milli\second}} = \SI{10}{\kilo\hertz}
    \]

    \[
    f_{1} = \frac{3}{\SI{0,2}{\milli\second}} = \SI{15}{\kilo\hertz}
    \]

    \[
    \Delta F = f_{1} - f_{0} = \SI{5}{\kilo\hertz}
    \]
\end{lösung}

\begin{figure}[H]
    \centering
    \includegraphics[width=0.8\textwidth]{images/ASK_FSK.pdf}
    \caption{Modulationsverfahren: ASK und FSK.}
    \label{fig:ook_modulation}
\end{figure}

\section{Antennen und Wellenlängen}

Eine Annäherung für eine Stabantenne ist der Hertzsche Dipol. Das Strahlungsmuster, dass sich für den hertzschen Dipol ergibt, ist in Abbildung \ref{3d_pattern} in 3D und in Abbildung \ref{2d_pattern} in 2D für die drei Hauptebenen in kartesischen Koordinaten dargestellt.

\begin{figure}[H]
    \centering
    \includegraphics[width=0.8\textwidth]{images/3d_radiation_pattern.pdf}
    \caption{3D Strahlungsmuster des Hertzschen Dipols.}
    \label{3d_pattern}
\end{figure}

\begin{aufgabe}
Zeichnen Sie das Strahlungsmuster der Stabantenne in 2D in den drei Hauptebenen der kartesischen Koordinaten, die in Abbildung \ref{2d_pattern} dargestellt sind.
\end{aufgabe}
\begin{figure}[H]
    \centering
    \includegraphics[width=.95\textwidth]{images/polar_radiation_pattern_empty.pdf}
    \caption{Hier soll das Strahlungsmuster der Stabantenne in 2D gezeichnet werden.}
    \label{2d_pattern}
\end{figure}

\begin{lösung}
    \centering
    \includegraphics[width=.9\textwidth]{images/polar_radiation_pattern.pdf}
    \captionof{figure}{2D Strahlungsmuster des Hertzschen Dipols.}
\end{lösung}




\section{Arduino und Transmitter Anschlüsse}
In diesem Abschnitt werden die Anschlüsse von Transmitter und Arduino beschrieben. Sie sind in Tabelle \ref{tab:arduino_transmitter_connections} zusammengefasst.

\begin{table}[H]
\centering
\noindent
\begin{tabular}{l l}
    \toprule
    \multicolumn{2}{c}{\textbf{Arduino}} \\
    \midrule
    \texttt{GND} & Masse \\
    \texttt{VCC} & Versorgung (5V/3.3V) \\
    \texttt{Digital I/O} & Digitale Ein-/Ausgänge \\
    \texttt{Analog In} & Analoge Eingänge \\
    \addlinespace % Fügt eine Leerzeile für eine visuelle Trennung ein
    \addlinespace
    \multicolumn{2}{c}{\textbf{Transmitter}} \\
    \midrule
    \texttt{GND} & Masse \\
    \texttt{VCC} & Versorgung \\
    \texttt{DATA} & Daten-Eingang \\
    \texttt{ANT} & Antenne \\
    \bottomrule

\end{tabular}
\caption{Anschlüsse des Arduino und des Transmitters}
\label{tab:arduino_transmitter_connections}
\end{table}


\begin{aufgabe}
    Verdrahten Sie die Bauteile, die Sie an Ihrem Arbeitsplatz finden und in Abbildung \ref{fig:arduino_433_schematic_skript} gezeigt sind sinnvoll miteinander.
\end{aufgabe}


\begin{figure}[H]
\centering
\includegraphics[width=0.8\textwidth]{images/Arduino_433_Sketch_Skript.pdf}
\caption{Schematic des Arduino mit Transmitter und Antenne.}
\label{fig:arduino_433_schematic_skript}
\end{figure}

\begin{lösung}

    \centering
    \includegraphics[width=0.8\textwidth]{images/Arduino_433_Sketch.pdf}
    \captionof{figure}{Schematic des Arduino mit $=\SI{433}{\mega\hertz}$ Transmitter, Antenne und korrekter Verdrahtung.}
\end{lösung}


\begin{aufgabe}
Ermitteln Sie auf dem RF-Transmitter die Trägerfrequenz. Entwerfen Sie eine Drahtantenne, die ein Viertel der Wellenlänge, also $\lambda/4$ lang ist. Verwenden Sie die Formel $\lambda = \frac{c}{f}$, wobei $c$ die Lichtgeschwindigkeit im Vakuum ($=\SI{3e8}{\meter\per\second}$) ist.
\end{aufgabe}
\karierteBox{5}

\begin{lösung}
Die Wellenlänge \(\lambda\) des Signals kann mit der Formel \(\lambda = \frac{c}{f}\) berechnet werden. Für \(f = 433\) MHz ergibt sich:

\[
\lambda = \frac{\SI{3e8}{\meter\per\second}}{\SI{433}{\mega\hertz}} \approx 0.693 \text{ m}
\]

Die Länge der Drahtantenne, die einem Viertel der Wellenlänge entspricht, ist daher:

\[
\frac{\lambda}{4} = \frac{0.693 \text{ m}}{4} \approx 0.173 \text{ m}
\]

Die Antenne sollte also etwa 17.3 cm lang sein.
\end{lösung}


% \section{Neuronale Netze}

Ein neuronales Netz besteht aus mehreren Schichten von Neuronen. Die wichtigsten Schichten sind die Eingabeschicht, eine oder mehrere versteckte Schichten (Hidden Layers) und die Ausgabeschicht.

\subsection*{Aufbau eines Neurons}

Jedes Neuron in einer Schicht ist mit den Neuronen der vorherigen Schicht verbunden. Jede dieser Verbindungen hat ein Gewicht \( w \), das die Stärke der Verbindung angibt. Zusätzlich hat jedes Neuron einen Bias \( b \), der als Schwellenwert dient.

Die Berechnung in einem Neuron erfolgt in zwei Schritten:

\begin{enumerate}
    \item \textbf{Lineare Kombination:} Zuerst wird eine gewichtete Summe der Eingaben gebildet:
    \[
    z = \sum_{i=1}^{n} w_i \cdot x_i + b
    \]
    Hierbei sind \( x_i \) die Eingaben, \( w_i \) die Gewichte und \( b \) der Bias.

    \item \textbf{Aktivierungsfunktion:} Anschließend wird das Ergebnis \( z \) durch eine Aktivierungsfunktion \( \sigma \) transformiert, um die Ausgabe des Neurons zu berechnen:
    \[
    h = g(z)
    \]

\end{enumerate}

\begin{figure}[h]
    \centering
    \begin{tikzpicture}[scale=1.2, transform shape]
        % Eingaben
        \node[circle, draw, minimum size=1cm] (I1) at (0, 2) {$x_1$};
        \node[circle, draw, minimum size=1cm] (I2) at (0, 0) {$x_2$};
        \node[circle, draw, minimum size=1cm] (I3) at (0, -2) {$x_3$};

        % Neuron
        \node[circle, draw, minimum size=1cm, fill=red!10] (N) at (3, 0) {$\sum$};

        % Aktivierungsfunktion Block
        \node[draw, minimum size=1cm, fill=yellow!10, align=center] (A) at (6, 0) {Aktivierungs-\\funktion $g(z)$};

        % Ausgang
        \node[circle, draw, minimum size=1cm, fill=blue!10] (O) at (9, 0) {$h$};

        % Verbindungen
        \draw[->] (I1) -- (N) node[midway, above] {$w_1$};
        \draw[->] (I2) -- (N) node[midway, above] {$w_2$};
        \draw[->] (I3) -- (N) node[midway, above] {$w_3$};
        \draw[->] (N) -- (A) node[midway, above] {$z$};
        \draw[->] (A) -- (O) node[midway, above] {$h$};

        % Bias
        \draw[->, dashed] (3, 1) -- (N) node[midway, right] {$b$};

    \end{tikzpicture}
    \caption{Einzelnes Neuron mit Gewichten, Bias und Aktivierungsfunktion.}
    \label{fig:single_neuron}
\end{figure}

\begin{figure}[h]
    \centering
    \begin{tikzpicture}[scale=1.2, transform shape]
        % Neuronen der Eingabeschicht
        \node[circle, draw, minimum size=1cm] (I1) at (-1, 2) {$x_1$};
        \node[circle, draw, minimum size=1cm] (I2) at (-1, 0) {$x_2$};
        \node[circle, draw, minimum size=1cm] (I3) at (-1, -2) {$x_3$};
        \node[draw=none] at (-1, 3) {Eingabeschicht};

        % Neuronen der versteckten Schicht
        \node[circle, draw, minimum size=1cm, fill=blue!10] (H1) at (3, 1) {$h_1$};
        \node[circle, draw, minimum size=1cm, fill=blue!10] (H2) at (3, -1) {$h_2$};
        \node[draw=none] at (3, 3) {Versteckte Schicht};

        % Neuronen der Ausgabeschicht
        \node[circle, draw, minimum size=1cm, fill=green!10] (O1) at (7, 0) {$y$};
        \node[draw=none] at (7, 3) {Ausgabeschicht};

        % Eingabeschicht zu versteckte Schicht
        \draw[->] (I1) -- (H1) node[midway, above] {$w_{11}$};
        \draw[->] (I1) -- (H2) node[midway, above] {$w_{12}$};
        \draw[->] (I2) -- (H1) node[midway, below] {$w_{21}$};
        \draw[->] (I2) -- (H2) node[midway, above] {$w_{22}$};
        \draw[->] (I3) -- (H1) node[midway, below] {$w_{31}$};
        \draw[->] (I3) -- (H2) node[midway, below] {$w_{32}$};

        % Versteckte Schicht zu Ausgabeschicht
        \draw[->] (H1) -- (O1) node[midway, above] {$w_{1}$};
        \draw[->] (H2) -- (O1) node[midway, above] {$w_{2}$};

        % gehe vom knoten nach oben und setze da den pfeil senkrecht nach unten mit der beschriftung
        \draw[->, dashed] (3, 2) -- (H1) node[midway, right] {$b_1$};
        \draw[->, dashed] (3, -2) -- (H2) node[midway, right] {$b_2$};
        \draw[->, dashed] (7, 1) -- (O1) node[midway, right] {$b_1$};
        
    \end{tikzpicture}
    \caption{Ein einfaches neuronales Netz mit einer Eingabeschicht, einer versteckten Schicht und einer Ausgabeschicht.}
    \label{fig:neural_network}
\end{figure}



Die Eingabeschicht besteht aus Neuronen, die die Daten ins Netz einspeisen. In der Grafik sind dies \( x_1 \), \( x_2 \) und \( x_3 \). Diese Neuronen nehmen die Rohdaten auf, die das Netz verarbeiten soll, dies können z.B. Sample Werte eines digitalen Signals sein.

Die versteckte Schicht (oder Hidden Layer) verarbeitet die Eingaben aus der Eingabeschicht. In der Grafik sind dies die Neuronen \( h_1 \) und \( h_2 \). Diese Schicht führt Berechnungen durch, um Muster und Merkmale in den Daten zu erkennen.

Die Ausgabeschicht liefert das Endergebnis des neuronalen Netzes. In der Grafik ist dies das Neuron \( y \). Dieses Neuron gibt z.B. die Wahrscheinlichkeit aus, dass ein Bild eine Katze zeigt.


\subsection{Aktivierungsfunktionen}

Aktivierungsfunktionen helfen einem neuronalen Netz, komplexe Muster in den Daten zu erkennen. Hier sind vier wichtige Aktivierungsfunktionen, die oft verwendet werden:

\begin{enumerate}
    \item \textbf{Sigmoid-Funktion:} 
    $$
    \sigma(z) = \frac{1}{1 + e^{-z}}
    $$
    Diese Funktion gibt Werte zwischen 0 und 1 aus.
    
    \item \textbf{Tanh-Funktion:}
    $$
    \tanh(z) = \frac{e^z - e^{-z}}{e^z + e^{-z}}
    $$
    Diese Funktion gibt Werte zwischen -1 und 1 aus.
    
    \item \textbf{ReLU-Funktion:}
    $$
    \mathrm{ReLU}(z) = \max(0, z)
    $$
    Diese Funktion gibt 0 aus, wenn \(z\) negativ ist, und \(z\), wenn \(z\) positiv ist.
    
    \item \textbf{Lineare Funktion:}
    $$
    g(z) = z
    $$
    Diese Funktion gibt den Eingabewert direkt aus.
\end{enumerate}

\begin{figure}[h]
    \centering
    \includegraphics[width=0.8\textwidth]{images/activation_functions.pdf}
    \caption{Verschiedene Aktivierungsfunktionen: Sigmoid, Tanh, ReLU und Linear.}
    \label{fig:activation_functions}
\end{figure}

\subsection{Training eines neuronalen Netzes}



\subsection{Training mit der ANT-GUI}

Die grafische Benutzeroberfläche ermöglicht es, ein neuronales Netz zu trainierren, dass die Samples eines OOK-Signals klassifiziert. Dazu wird ein Datensatz benötigt, der die Samples des Signals enthält. Dieser Datensatz wird in zwei Teile aufgeteilt: einen Trainingsdatensatz und einen Testdatensatz. Der Trainingsdatensatz wird verwendet, um das neuronale Netz zu trainieren, während der Testdatensatz verwendet wird, um die Genauigkeit des Netzes zu überprüfen.
Der Startbildschirm der Oberfläche ist in Abbildung \ref{fig:ant_gui} zu sehen. Ein OOK Symbol besteht aus 100 Samles, was im Netzwerk als 100 Eingaben interpretiert wird. 


\begin{figure}[h]
    \centering
    \includegraphics[width=0.75\textwidth]{images/GUI_start.png}
    \caption{Die ANT-GUI mit einem trainierten neuronalen Netz.}
    \label{fig:ant_gui}
\end{figure}

\begin{aufgabe}
    Trainiere ein neuronales Netz, bei dem die Genauigkeit mindesten 90\% beträgt und speichere das Modell ab.
\end{aufgabe}

\begin{lösung}
    Die Ergebnisse hängen nicht von der Tiefe und Komplexität des Netzwerks ab, sondern vor allem von der gewählten Kostenfunktion und den Aktivierungsfunktionen.
\end{lösung}

Ein beispielhaftes Netzwerk, wie es in der GUI angezeigt wird, ist in Abbildung \ref{fig:gui_network} zu sehen.

\begin{figure}[h]
    \centering
    \includegraphics[width=0.8\textwidth]{images/GUI_network.png}
    \caption{Ein beispielhaftes neuronales Netz in der ANT-GUI.}
    \label{fig:gui_network}
\end{figure}

Das Netwerk besitzt 3 versteckte Schichten mit je 5 Neuronen pro Schicht. Die Eingabeschicht hat 100 Neuronen, die die Samples des OOK-Signals repräsentieren. Die Ausgabeschicht hat 1 Neuron, das die Wahrscheinlichkeit ausgibt, dass das Signal eine 1 bzw. eine 0 ist.


\section{OOK Erkennung mit einem neuronalen Netz in Gnuradio}

GNU Radio ist eine Open-Source-Software, die es ermöglicht, Radiosignale auf einem Computer zu verarbeiten. Mit GNU Radio kann man verschiedene Funksignale empfangen, analysieren und senden, indem man Module und Bausteine zu einem Signalverarbeitungssystem zusammenfügt.

Ein RTL-SDR (Software Defined Radio) ist ein günstiger USB-Stick, der ursprünglich als TV-Tuner entwickelt wurde, aber auch verwendet werden kann, um Funksignale aus der Umgebung zu empfangen. Mit diesen beiden Tools können wir ein OOK-Signal empfangen und mit einem konventioneller Signalverarbeitung oder einem neuronalen Netz erkennen.







% \include{fortgeschrittenes_modul}


\end{document}
