
\documentclass{article}
\usepackage{geometry}
\usepackage{amsmath,amssymb}
\usepackage{graphicx}
\usepackage{hyperref}

\title{Modulare Versuchsreihe zur Nachrichtentechnik mit Arduino und RTL-SDR}
\author{}
\date{\today}

\begin{document}

\maketitle

\section{Grundlagen}
\begin{itemize}
    \item Eine einfache Erklärung, wie Informationen drahtlos übertragen werden, einschließlich grundlegender Konzepte wie Frequenzen, Amplitude, Modulationen (evtl. mit GNURadio Signalerzeugung?)
    \item Verständliche Erläuterung der OOK-Modulation, unterstützt durch anschauliche Beispiele.
    \item Anleitung zum Aufbau eines einfachen Senders mit einem Arduino und einem 433 MHz Modul, inklusive eines einfachen Codes, um ein Signal zu senden.
    \item Signal Empfangen und Visualisieren mit einem RTL-SDR und GNURadio (alternativ mit zweitem Arduino als Empfänger)
    \item \textbf{Umfang: 2 Stunden}
\end{itemize}

\section{Erweitert}
\begin{itemize}
    \item Ein Überblick über einfache digitale Signalverarbeitung im Basisband. Sample Theorie (Shannon-Nyquist)
    \item Entwurf von Code für eine Einfache OOK-Demodulation (Schwellenwertentscheidung)
    \item \textbf{Umfang: 1-2 Stunden}
\end{itemize}

\section{Fortgeschritten}
\begin{itemize}
    \item Eine einfache Einführung in das maschinelle Lernen 
    \item Integration des fertigen Modells mit verschieden umfangreichen Trainingssets
    \item Vergleich der Genauigkeit der verschiedenen Modelle und der selbst implementierten Schwellenwertentscheidung
    \item \textbf{Umfang: 1-2 Stunden}
\end{itemize}


\end{document}
